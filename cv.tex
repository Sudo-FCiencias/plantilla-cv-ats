% Plantilla profesional de CV de una página optimizada para
% sistemas de seguimiento de candidatos (ATS-friendly). La
% es que asi como edites este archivo para mandar en digital,
% puedas imprimirlo en tus entrevistas presenciales. 
% 
% Instrucciones:
% - Reemplaza todos los valores de ejemplo con tu información (nombres de usuario, nombres, etc.)
% - Comenta las secciones que no desees incluir (Proyectos, etc.)
% - Si lo necesitas, modifica config.text para ajustar según tus necesidades.

\documentclass[a4paper, 10pt]{article}
% Este archivo contiene todas las configuraciones, paquetes y comandos personalizados necesarios para la plantilla de CV.
% Compilador: pdfLaTeX
% Versión de TeX: 2024

% Configuración de página completa (márgenes mínimos)
\usepackage[empty]{fullpage}
% Personalización de formato de títulos de secciones
\usepackage{titlesec}
% Símbolos adicionales (teléfono, email, etc.)
\usepackage{marvosym}
\usepackage[hidelinks]{hyperref}
\usepackage{enumitem}
\usepackage{fancyhdr}
\usepackage[english]{babel}
\usepackage{tabularx}
% Íconos de redes sociales (GitHub, LinkedIn, etc.)
\usepackage{fontawesome5}
\usepackage{multicol}
% Elimina el espacio entre columnas
\setlength{\columnsep}{0pt}
% Ajusta el espacio vertical entre columnas
\setlength{\multicolsep}{2pt plus 1pt minus 0.5pt}
% Usa Roboto como fuente sans-serif por defecto
% Otras opciones: helvet, arial, etc.
\usepackage[sfdefault]{roboto}
% Estilo de página fancy (con encabezados personalizados)
\pagestyle{fancy}
\fancyhf{} % Limpia encabezados y pies de página

% Elimina las líneas horizontales de encabezado y pie de página
\renewcommand{\headrulewidth}{0pt}
\renewcommand{\footrulewidth}{0pt}

% Estos valores controlan el espacio alrededor del contenido del CV
% Ajústalos según tus preferencias

\addtolength{\oddsidemargin}{-0.6in} % Margen izquierdo (páginas impares)
\addtolength{\evensidemargin}{-0.6in} % Margen izquierdo (páginas pares)
\addtolength{\textwidth}{1.2in} % Ancho del texto
\addtolength{\topmargin}{-.6in} % Margen superior
\addtolength{\textheight}{1.0in} % Altura del texto

% Define cómo se verán los títulos de las secciones
% \large = tamaño grande
% \bfseries = negrita
% \raggedright = alineado a la izquierda
% \titlerule = línea horizontal debajo del título

\titleformat{\section}{ \large\bfseries\scshape\raggedright }{}{0pt}{}[\titlerule]

% Asegura que el PDF resultante sea "machine readable" (ATS-friendly)
\pdfgentounicode=1

% Comando para crear un ítem individual en listas
% Uso: \resumeItem{Descripción del logro o responsabilidad}
\newcommand{\cvItem}[1]{ \item\small{#1} }

% Comando para crear un subencabezado de experiencia o educación
% Parámetros:
%   #1 = Título del puesto o institución (en negrita)
%   #2 = Fecha (alineada a la derecha)
%   #3 = Subtítulo (empresa/ubicación, en cursiva)
%   #4 = Período adicional (en cursiva, alineado a la derecha)
% Uso:
% \resumeSubheading
%   {Ingeniero de Software}{Enero 2020 - Presente}
%   {Empresa XYZ}{Ciudad, País}
\newcommand{\cvSubtitulo}[4]{
\vspace{-2pt}
\item
\begin{tabular*}{1.0\textwidth}[t]{l@{\extracolsep{\fill}}r}
  \textbf{#1} & #2          \\
  \textit{#3} & \textit{#4} \\
\end{tabular*}
\vspace{-7pt}
}

% Redefine el símbolo de las listas de segundo nivel (sub-bullets)
% Usa un bullet point más pequeño
\renewcommand{\labelitemii}{$\vcenter{\hbox{\tiny$\bullet$}}$}

% Comando para iniciar una lista de subsecciones
% Ajusta el margen izquierdo y elimina las etiquetas por defecto
\newcommand{\ListaSubtitulo}{ \begin{itemize}[leftmargin=0.1in, label={}] }

% Comando para cerrar la lista de subsecciones
\newcommand{\ListaSubtituloFin}{ \end{itemize} }

\begin{document}
  %==================================================================
  % Encabezado
  % - Aqui incluyes tu información de contacto.
  % - Reemplaza los valores con tu información personal
  % - Sigue los formatos para saber que debes llenar.

  \begin{center}
    % Nombre completo en grande y negrita. No uses pseudónimos.
    \textbf{\LARGE Tu Nombre Completo} \\
    \vspace{3mm}

    % Información de contacto con íconos
    \small \faPhone \hspace{1mm} 5544778833 $|$ \hspace{1mm} \faEnvelope \hspace{1mm}
    \href{mailto:correo@mail.com}{correo@mail.com} $|$ \hspace{1mm} \faLinkedin \hspace{1mm}
    \href{https://www.linkedin.com/in/username/}{https://www.linkedin.com/in/username/}
    $|$ \hspace{1mm} \faGithub \hspace{1mm} \href{https://github.com/username}{https://github.com/username}\\
    % Si tienes portafolio y/o sitio web personal, añadelo aquí
    $|$ \hspace{1mm} \faGlobe \hspace{1mm} \href{https://tuportafolio.com}{tuportafolio.com}
  \end{center}

  %==================================================================
  % Resumen Profesional
  % - Un párrafo breve (3-4 líneas) que resuma tu perfil profesional
  % - Incluye: título profesional, años de experiencia, habilidades clave, objetivos
  % etc.

  \section{Resumen Profesional}
  Escribe aquí tu resumen profesional. Ejemplo: "Ingeniero de Software con 3+
  años de experiencia en desarrollo full-stack. Especializado en Python,
  JavaScript y arquitecturas cloud. Apasionado por crear soluciones escalables y liderar
  equipos ágiles. Busco contribuir en proyectos innovadores de tecnología."

  %==================================================================
  % Habilidades Blandas y Técnicas
  % - Organiza tus habilidades en categorías usando múltiples columnas.
  % - Donde sientas que debe cortarse la linea, ocupar // es suficiente.

  \section{Habilidades}
  \ListaSubtitulo
  \begin{multicols}{2}
    {
    % Columna 1: Habilidades técnicas
    \cvItem{\textbf{Lenguajes de Programación}: Ej: Python, Java, JavaScript, C++} \cvItem{\textbf{Frameworks \& Librerías}: Ej: React, Django, Spring Boot, Node.js} \cvItem{\textbf{Bases de Datos}: Ej: MySQL, PostgreSQL, MongoDB, Redis} \cvItem{\textbf{Herramientas \& DevOps}: Ej: Git, Docker, Kubernetes, AWS} }
    {
    % Columna 2: Idiomas y habilidades blandas
    \cvItem{\textbf{Idiomas}: Ej: Español (Nativo), Inglés (Avanzado - C1), Francés (Intermedio - B2)} \cvItem{\textbf{Habilidades Blandas}: Ej: Liderazgo de equipos, Comunicación técnica, Resolución de problemas, Metodologías Ágiles (Scrum), Gestión del tiempo} }
  \end{multicols}
  \ListaSubtituloFin

  %==================================================================
  % Educación
  % - Lista tus títulos académicos en orden cronológico inverso (los que tengan
  % más peso primero, como un doctorado o una licenciatura).
  % - Si aún estå en curso, puedes poner una fecha aproximada de término.

  \section{Educación}
  \ListaSubtitulo

  \cvSubtitulo {Universidad ABC}{Ciudad, País} {Licenciatura en ABCDEFG}{Mes 20XX - Mes 20XX}
  % Opcional: Agrega logros académicos, promedio (si es alto), honores, etc.
  % \ListaSubtitulo
  %     \cvItem{\textbullet\ Promedio: 9.5/10.0}
  %     \cvItem{\textbullet\ Mención honorífica}
  % \ListaSubtituloFin

  % Agrega más títulos si los tienes (maestría, doctorado, etc.)

  \ListaSubtituloFin

  %=================================================================
  % Proyectos (OpcionaL)
  % - Usa esta sección si quieres incluir proyectos personales o académicos
  % - Útil para estudiantes o personas con poca experiencia laboral

  \section{Proyectos Destacados}
  \ListaSubtitulo

  \cvSubtitulo
  {Nombre del Proyecto $|$ \faGithub \href{https://github.com/username/repo}{ Repositorio en GitHub}}
  {Mes 20XX} {Descripción en una linea del Proyecto}{Lista de tecnologías usadas.}
  \ListaSubtitulo \cvItem{\textbullet\ Describe qué hace el proyecto y su impacto.}
  \cvItem{\textbullet\ Tecnologías o metodologías destacadas que usaste.} \cvItem{\textbullet\ Resultados o métricas si las hay.}
  \ListaSubtituloFin

  \ListaSubtituloFin

  %=================================================================
  % Experiencia Laboral
  % - Lista tu experiencia laboral en orden cronológico inverso (más reciente primero)
  % - Para cada puesto, incluye 2-4 logros cuantificables

  \section{Experiencia Profesional}
  \ListaSubtitulo

  % Primer trabajo (más reciente)
  \cvSubtitulo {Nombre del Puesto}{Mes 20XX - Presente} {Empresa ABC}{Ciudad, País}
  \ListaSubtitulo \cvItem{\textbullet\ Describe un logro importante con métricas. Ej: "Desarrollé una API REST que redujo el tiempo de respuesta en un 40\%, mejorando la experiencia de 10,000+ usuarios"}
  \cvItem{\textbullet\ Otro logro. Ej: "Lideré un equipo de 5 desarrolladores en la implementación de una arquitectura de microservicios, resultando en una reducción del 30\% en costos de infraestructura"}
  \cvItem{\textbullet\ Otro logro o responsabilidad relevante} \ListaSubtituloFin

  % Puedes agregar más trabajos copiando el formato anterior

  \ListaSubtituloFin

  %==================================================================
  % Actividades Extracurriculares (Opcional)
  % - Incluye voluntariado, liderazgo estudiantil, comunidades tech, etc.

  \section{Actividades Extracurriculares}
  \ListaSubtitulo

  \cvSubtitulo {Puesto}{Mayo 2024 - Enero 2026} {Comunidad Tech ABC}{Ciudad, País}
  \ListaSubtitulo \cvItem{\textbullet\ Describe tu rol y contribuciones} \cvItem{\textbullet\ Logros o impacto medible si es posible}
  \ListaSubtituloFin

  % Puedes agregar más actividades

  \ListaSubtituloFin

  %==================================================================
  % Certificaciones / Credenciales.
  % - Lista certificaciones profesionales relevantes con enlaces que lleven a la
  % misma.
  % - También añade las "badges" que tengas por completar algun curso.

  \section{Certificaciones \& Credenciales}
  \ListaSubtitulo
  \begin{multicols}{2}
    { \cvItem{\textbf{Certificación en XYZ} \href{example.com}{$|$ Ver Credencial}} { \cvItem{\textbf{Completitud de Curso} $|$ ID: 000000}}
    % Agrega más certificaciones según necesites
    }
  \end{multicols}
  \ListaSubtituloFin
\end{document}

%==================================================================
% Notas y Tips Finales.
%
% - Mantén tu CV en 1-2 páginas máximo
% - Usa verbos fuertes al inicio de cada bullet point:
%   Desarrollé, Implementé, Lideré, Optimicé, Diseñé, Automaticé...
% - Siempre que sea posible, incluye números:
%    - "Mejoré el rendimiento en un 40%"
%    - "Gestioné un equipo de 5 personas"
%    - "Reduje costos en $10,000 anuales"
% - Ajusta tu CV para cada solicitud de trabajo, no uses el mismo para todo.       
%   De ésta manera destacas las habilidades y experiencias más relevantes para el  %   puesto.
% - Usa palabras clave del anuncio de trabajo para que tu CV pase los sistemas
%   automatizados de selección.
% - Verifica ortografía y gramática. Un error puede descalificarte
% - Mantén un formato consistente (fechas, mayúsculas, puntos)
