% Este archivo contiene todas las configuraciones, paquetes y comandos personalizados necesarios para la plantilla de CV.
% Compilador: pdfLaTeX
% Versión de TeX: 2024

% Configuración de página completa (márgenes mínimos)
\usepackage[empty]{fullpage}
% Personalización de formato de títulos de secciones
\usepackage{titlesec}
% Símbolos adicionales (teléfono, email, etc.)
\usepackage{marvosym}
\usepackage[hidelinks]{hyperref}
\usepackage{enumitem}
\usepackage{fancyhdr}
\usepackage[english]{babel}
\usepackage{tabularx}
% Íconos de redes sociales (GitHub, LinkedIn, etc.)
\usepackage{fontawesome5}
\usepackage{multicol}
% Elimina el espacio entre columnas
\setlength{\columnsep}{0pt}
% Ajusta el espacio vertical entre columnas
\setlength{\multicolsep}{2pt plus 1pt minus 0.5pt}
% Usa Roboto como fuente sans-serif por defecto
% Otras opciones: helvet, arial, etc.
\usepackage[sfdefault]{roboto}
% Estilo de página fancy (con encabezados personalizados)
\pagestyle{fancy}
\fancyhf{} % Limpia encabezados y pies de página

% Elimina las líneas horizontales de encabezado y pie de página
\renewcommand{\headrulewidth}{0pt}
\renewcommand{\footrulewidth}{0pt}

% Estos valores controlan el espacio alrededor del contenido del CV
% Ajústalos según tus preferencias

\addtolength{\oddsidemargin}{-0.6in} % Margen izquierdo (páginas impares)
\addtolength{\evensidemargin}{-0.6in} % Margen izquierdo (páginas pares)
\addtolength{\textwidth}{1.2in} % Ancho del texto
\addtolength{\topmargin}{-.6in} % Margen superior
\addtolength{\textheight}{1.0in} % Altura del texto

% Define cómo se verán los títulos de las secciones
% \large = tamaño grande
% \bfseries = negrita
% \raggedright = alineado a la izquierda
% \titlerule = línea horizontal debajo del título

\titleformat{\section}{ \large\bfseries\scshape\raggedright }{}{0pt}{}[\titlerule]

% Asegura que el PDF resultante sea "machine readable" (ATS-friendly)
\pdfgentounicode=1

% Comando para crear un ítem individual en listas
% Uso: \resumeItem{Descripción del logro o responsabilidad}
\newcommand{\cvItem}[1]{ \item\small{#1} }

% Comando para crear un subencabezado de experiencia o educación
% Parámetros:
%   #1 = Título del puesto o institución (en negrita)
%   #2 = Fecha (alineada a la derecha)
%   #3 = Subtítulo (empresa/ubicación, en cursiva)
%   #4 = Período adicional (en cursiva, alineado a la derecha)
% Uso:
% \resumeSubheading
%   {Ingeniero de Software}{Enero 2020 - Presente}
%   {Empresa XYZ}{Ciudad, País}
\newcommand{\cvSubtitulo}[4]{
\vspace{-2pt}
\item
\begin{tabular*}{1.0\textwidth}[t]{l@{\extracolsep{\fill}}r}
  \textbf{#1} & #2          \\
  \textit{#3} & \textit{#4} \\
\end{tabular*}
\vspace{-7pt}
}

% Redefine el símbolo de las listas de segundo nivel (sub-bullets)
% Usa un bullet point más pequeño
\renewcommand{\labelitemii}{$\vcenter{\hbox{\tiny$\bullet$}}$}

% Comando para iniciar una lista de subsecciones
% Ajusta el margen izquierdo y elimina las etiquetas por defecto
\newcommand{\ListaSubtitulo}{ \begin{itemize}[leftmargin=0.1in, label={}] }

% Comando para cerrar la lista de subsecciones
\newcommand{\ListaSubtituloFin}{ \end{itemize} }